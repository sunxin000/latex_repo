\documentclass[UTF8,titlepage]{ctexart}

\title{DPI Homework 4}
\author{孙鑫}
\date{\today}
\usepackage{amsmath}

\newcommand{\suminf}{\sum_{n=-\infty}^{\infty}}
\begin{document}
\maketitle
\section*{4.2}
\setlength{\parindent}{0pt}Show that $\tilde{F}(\mu)$ in Eq. (4.4-2) is infinitely periodic in both directions, with period $1/\Delta T$

\textbf{SOLUTION:}

先证明:
\begin{align*}
    \tilde{F}(\mu+k[1/\Delta T])=\tilde{F}(\mu)
\end{align*}
证明如下:

根据公式4.3-5
\begin{align*}
    \tilde{F}(\mu+k[1/\Delta T]) & =\frac{1}{\Delta T}\sum_{n=-\infty}^{\infty}F(\mu+\frac{k}{\Delta T}-\frac{n}{\Delta T})\\
        & = \frac{1}{\Delta T}\sum_{n=-\infty}^{\infty}F(\mu+\frac{k-n}{\Delta T})\\
        & = \frac{1}{\Delta T}\sum_{m=-\infty}^{\infty}F(\mu-\frac{m}{\Delta T})\\
        & = \tilde{F}(\mu)
\end{align*}

再证明:
\begin{displaymath}
    \tilde{F}(\mu+k\Delta T)=\tilde{F}(\mu)
\end{displaymath}

证明如下:

根据公式4.4-2
\begin{align*}
    \tilde{F}(\mu+k\Delta T) &=\suminf f_ne^{-j2\pi(\mu+k/\Delta T)n\Delta T}\\
    &=\suminf f_ne^{-j2\pi\mu n\Delta T}e^{-j2\pi kn}\\
    &=\suminf f_ne^{-j2\pi\mu n\Delta T}\\
    &=\tilde{F}(\mu)
\end{align*}
综上,$\tilde{F}(\mu)$ 在两个方向上具有无穷周期性

\end{document}